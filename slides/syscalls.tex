\begin{frame}

\frametitle{System Calls (1/3)}

\vspace{\fill}

\begin{center}

An operating system mediates the users' access \\ to underlying physical
devices.

\end{center}

\begin{center}

User programs issue system calls \\ to get things done.

\end{center}

\vspace{\fill}

\end{frame}


\begin{frame}[fragile]

\frametitle{System Calls (2/3): What's To a System Call?}

\vspace{\fill}

A system call is:

\begin{itemize}

\item A system call number (i.e., handler has a switch case).

\item A handful of register-resident arguments.

\item A register-resident result (more outcomes on next slide).

\end{itemize}

Examples:

\begin{lstlisting}
#include <unistd.h>
ssize_t read(int fd, void *buf, size_t count);
ssize_t write(int fd, const void *buf, size_t count);
\end{lstlisting}

\vspace{\fill}

\end{frame}


\begin{frame}

\frametitle{System Calls (3/3): More Examples}

\vspace{\fill}

What could possibly happen?

\begin{itemize}

\item Opening, reading, writing, or closing a file.

\item Mounting a file system.

\item Spawning a subprocess.

\item Killing another process.

\item System halts.

\item Anything!

\end{itemize}

\vspace{\fill}

\end{frame}
