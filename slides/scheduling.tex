\begin{frame}

\begin{center}

\Huge \textbf{Scheduling}

\end{center}

\end{frame}


\begin{frame}

\frametitle{Scheduling Policies}

\vspace{\fill}

\begin{center}

% \item An operating system provides the illusion that a task runs alone on a
% given processor. This is achieved by occasionally switching between which
% tasks are run.

In the context of one processing core, \\ if more than one task is ready to
run, \\ which should get to go first?

\end{center}

\vspace{\fill}

\end{frame}

\begin{frame}

\frametitle{Scheduling Metrics}

\begin{center}

We need measures for comparing scheduling policies.

\end{center}

\end{frame}


\begin{frame}

\frametitle{Average Turnaround Time}

$$\frac{\sum_{i=1}^n T^i_{\text{turnaround}}}{n}$$

where,

$$T^i_{\text{turnaround}} = T^i_{\text{completion}} - T^i_{\text{arrival}}$$

\begin{itemize}

% \item Due to assumption (2), we can assume $T_{\text{arrival}} = 0$.

\item This is a \emph{performance} metric, not a \emph{fairness} metric.

\begin{itemize}

\item We can manipulate averages at the cost of fairness.

\end{itemize}

\end{itemize}

\end{frame}


\begin{frame}

\frametitle{Average Response Time}

$$\frac{\sum_{i=1}^n T^i_{\text{response}}}{n}$$

where,

$$T^i_{\text{response}} = T^i_{\text{firstrun}} - T^i_{\text{arrival}}$$

\begin{itemize}

% \item Due to assumption (2), we can assume $T_{\text{arrival}} = 0$.

\item This is \emph{still} not a fairness metric.

\begin{itemize}

\item We can manipulate averages at the cost of fairness.

\end{itemize}

\item This is an \emph{interactive} performance metric.

\end{itemize}

\end{frame}


\begin{frame}

\frametitle{Summing Up: Points of Measure}

\vspace{\fill}

Task parameters indpendent of the scheduler:

\begin{itemize}

\item[1.] The task arrival time, $T_{\text{arrival}}$.

\item[2.] The task execution time, $T_{\text{execution}}$.

\end{itemize}

The following depends on the scheduler:

\begin{itemize}

\item[3.] The time a task is first run, $T_{\text{firstrun}}$.

\item[4.] The time a task completes, $T_{\text{completion}}$.

\end{itemize}

\vspace{\fill}

\begin{center}

(1) and (2), can play a varying role in (3) and (4), \\ depending on the
scheduling policy.

\end{center}

% \vspace{\fill}

% \begin{itemize}

% \item Due to assumption (2), we can assume $T_{\text{arrival}} = 0$.

% \end{itemize}

\end{frame}


\begin{frame}

\frametitle{Simplifying Assumptions}

Let us begin with some unrealistic assumptions:

\begin{itemize}

\item[1.] Each task runs for the same, fixed amount of time.

\item[2.] All tasks arrive at the same time.

\item[3.] Once started, a task runs to completion.

\item[4.] All tasks only use the CPU (i.e., perform no I/O)

% \item[5.] The wall-clock running time of each task is known.

\end{itemize}

\vspace{\fill}

Some formal consequences (more along the way):

\begin{itemize}

\item[a.] There is a fixed number of $n$ tasks in the system.

\end{itemize}

\end{frame}


\begin{frame}

\frametitle{First-In, First-Out (FIFO)}

\begin{itemize}

\item Sometimes also called First-Come, First-Serve (FCFS).

\end{itemize}

\end{frame}


\begin{frame}

\frametitle{Shortest Job First (SJF)}

\end{frame}


\begin{frame}

\frametitle{Preemptive Scheduling}

\vspace{\fill}

\begin{center}

A non-preemptive scheduler runs a task to completion.

\end{center}

\begin{center}

A preemptive scheduler interrupts a task \\ whenever there are better things to
do.

\end{center}

\vspace{\fill}

\end{frame}

\begin{frame}

\frametitle{Shortest Time to Completion First (STCF)}

\end{frame}


\begin{frame}

\frametitle{Round Robin (RR)}

\end{frame}


\begin{frame}

\frametitle{I/O Scheduling}

\begin{itemize}

\item I/O operations may again take a varying amount of time.

\item Some systems have separate I/O, CPU time schedulers.

\end{itemize}

\end{frame}


\begin{frame}

\frametitle{Multi-Level Feedback Queues (MLFQ)}

\end{frame}

