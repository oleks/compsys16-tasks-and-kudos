\begin{frame}
    \frametitle{\kudos{} workflow}

    \begin{center}
        \kudos{} source overview
    \end{center}
    \vspace{\fill}

    \begin{itemize}
        \item \textbf{kudos} The \kudos{} kernel code
            \begin{itemize}
                \item \textbf{drivers} Driver subsystem
                \item \textbf{fs} File-system subsystem
                \item \textbf{init} Initialization (aka. boot) subsystem
                \item \textbf{kernel} Core elements subsystem
                \item \textbf{lib} Helper functions subsystem
                \item \textbf{util} Utility to create TFS disk images
                \item \textbf{vm} Virtual memory subsystem
            \end{itemize}
        \item \textbf{userland} Userland programs for \kudos{}
            \begin{enumerate}
                \item \textbf{halt.c} Calls the halt sys-call
            \end{enumerate}
        \item \textbf{run\_qemu.sh} Runs \kudos{}
    \end{itemize}

    \vspace{\fill}
\end{frame}

\begin{frame}
    \frametitle{\kudos{} workflow}

    \begin{center}
        Tools
    \end{center}
    \vspace{\fill}

    \begin{itemize}
        \item \textbf{Qemu} Virtual machine
        \item \textbf{grub-mkrescue/xorriso} Create a bootable ISO file
        \item \textbf{kudos/utils/tfstool} Create tfs (trivial filesystem) image, and transfer userland programs to the disk image
    \end{itemize}
\end{frame}

\begin{frame}

\frametitle{Running \kudos{}}

\vspace{\fill}
\begin{center}
    When using ssh and the VirtualBox image
\end{center}
\begin{itemize}
    \item \textbf{Linux and Mac users} Connect to ssh with \lstinline{ssh archimedes@localhost -p 1337 -X}
    \item \textbf{Windows users} Start Xming before connecting with Putty, and remember to enable X-forward in Putty
\end{itemize}

\vspace{\fill}

\end{frame}
