\begin{frame}

\frametitle{Multi-Level Feedback Queues (MLFQ): Rules}

\begin{itemize}

\item Maintain multiple RR queues with varying time slices.

\item So there are multiple ``levels'' where a task may be.

\item Lower levels have shorter timeslices and higher levels.

%\item Short-

\end{itemize}

\end{frame}

\begin{frame}

\frametitle{Multi-Level Feedback Queues (MLFQ): Rules}

\begin{description}

\item [Rule 1] If $\text{Level}(A) < \text{Level}(B)$, $A$ runs ($B$ doesn't).

\item [Rule 2] If $\text{Level}(A) = \text{Level}(B)$, $A$ \& $B$ run in RR. \\

\item [Rule 3] When a job enters the system, it is placed in the lowest queue.

\item [Rule 4] Once a job uses up its time-slice in the RR-scheme (see Rule 2),
it is moved one level up, unless the job finished, or it is already on the
highest queue.

\item [Rule 5] After a time period of $x$ ticks, move all the jobs in the
system to the lowest queue.

\end{description}

\begin{center}

(Demo left as an exercise.)

\end{center}

\end{frame}
